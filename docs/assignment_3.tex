%%%%%%%%%%%%%%%%%%%%%%%%%%%%%%%%%%%%%%%%%
% Short Sectioned Assignment
% LaTeX Template
% Version 1.0 (5/5/12)
%
% This template has been downloaded from:
% http://www.LaTeXTemplates.com
%
% Original author:
% Frits Wenneker (http://www.howtotex.com)
%
% License:
% CC BY-NC-SA 3.0 (http://creativecommons.org/licenses/by-nc-sa/3.0/)
%
%%%%%%%%%%%%%%%%%%%%%%%%%%%%%%%%%%%%%%%%%

%----------------------------------------------------------------------------------------
%	PACKAGES AND OTHER DOCUMENT CONFIGURATIONS
%----------------------------------------------------------------------------------------

\documentclass[paper=a4, fontsize=11pt]{scrartcl} % A4 paper and 11pt font size

%\usepackage[T1]{fontenc} % Use 8-bit encoding that has 256 glyphs
%\usepackage{fourier} % Use the Adobe Utopia font for the document - comment this line to return to the LaTeX default
\usepackage[english]{babel} % English language/hyphenation
\usepackage{amsmath,amsfonts,amsthm} % Math packages

\usepackage{lipsum} % Used for inserting dummy 'Lorem ipsum' text into the template

\usepackage{sectsty} % Allows customizing section commands

\usepackage{mystyle}
\usepackage{cancel}

\allsectionsfont{\centering \normalfont\scshape} % Make all sections centered, the default font and small caps

\usepackage{fancyhdr} % Custom headers and footers
\pagestyle{fancyplain} % Makes all pages in the document conform to the custom headers and footers
\fancyhead{} % No page header - if you want one, create it in the same way as the footers below
\fancyfoot[L]{} % Empty left footer
\fancyfoot[C]{} % Empty center footer
\fancyfoot[R]{\thepage} % Page numbering for right footer
\renewcommand{\headrulewidth}{0pt} % Remove header underlines
\renewcommand{\footrulewidth}{0pt} % Remove footer underlines
\setlength{\headheight}{13.6pt} % Customize the height of the header

\numberwithin{equation}{section} % Number equations within sections (i.e. 1.1, 1.2, 2.1, 2.2 instead of 1, 2, 3, 4)
\numberwithin{figure}{section} % Number figures within sections (i.e. 1.1, 1.2, 2.1, 2.2 instead of 1, 2, 3, 4)
\numberwithin{table}{section} % Number tables within sections (i.e. 1.1, 1.2, 2.1, 2.2 instead of 1, 2, 3, 4)

\setlength\parindent{0pt} % Removes all indentation from paragraphs - comment this line for an assignment with lots of text

%----------------------------------------------------------------------------------------
%	TITLE SECTION
%----------------------------------------------------------------------------------------

\newcommand{\horrule}[1]{\rule{\linewidth}{#1}} % Create horizontal rule command with 1 argument of height

\title{	
    \normalfont \normalsize 
%\textsc{university, school or department name} \\ [25pt] % Your university, school and/or department name(s)
    \horrule{0.5pt} \\[0.4cm] % Thin top horizontal rule
    \huge Zero-Mean Pressure \\ % The assignment title
    \horrule{2pt} \\[0.5cm] % Thick bottom horizontal rule
}

\author{Will Frey} % Your name

\date{\normalsize\today} % Today's date or a custom date

\begin{document}

\maketitle % Print the title

%----------------------------------------------------------------------------------------
%	PROBLEM 1
%----------------------------------------------------------------------------------------

\section{Zero-Mean Pressure Calculation}

Since $\del p$ isn't distinguishable from $\del (p + c)$ for $c$ a constant,
I'm going to investigate since I'm fairly certain what I did in my
implementation was correct.

By the divergence theorem, since $\vect{v} = 0$ on the boundary:
\begin{equation*}
    \int \del \cdot \left( \left( p + c \right) \vect{v} \right) d\Omega
    =  \oint \left( p + c \right)\vect{v} \cdot \vect{\hat{n}} d\Gamma = 0
\end{equation*}

\begin{align*}
    \begin{split}
        \int \del \cdot \left( \left( p + c \right) \vect{v} \right) d\Omega
        %
        &=  \int \del \cdot \left( p \vect{v} \right) d\Omega +
        \int \del \cdot \left( c \vect{v} \right) d\Omega \\
        %
        &=  \int \del p \vect{v} d\Omega +
        \int p \del \cdot \vect{v} d\Omega +
        \cancelto{0}{\int \vect{v} \del c d\Omega} +
        \int c \del \cdot \vect{v} d\Omega \\
        %
        &=  \int \del p \vect{v} d\Omega +
        \int p \del \cdot \vect{v} d\Omega +
        c \int \del \cdot \vect{v} d\Omega
    \end{split}
\end{align*}

\begin{equation*}
    \int \del p \vect{v} d\Omega
    =  -\int p \del \cdot \vect{v} d\Omega -
    c \int \del \cdot \vect{v} d\Omega 
\end{equation*}

If I assert that $\int \del p \, \vect{v} \, d\Omega \equiv - \int p \del \cdot
\vect{v}  d\Omega$ in my implementation, aren't I then fixing $c = 0$ and
calculating a unique pressure? That's what I did in my implementation.
\end{document}
